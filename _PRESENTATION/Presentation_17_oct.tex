\documentclass{beamer}

\usepackage[T1]{fontenc}
\usepackage[utf8]{inputenc}
\usepackage[english]{babel}
\usepackage{lmodern}
\usepackage{tikz}
\usetikzlibrary{positioning}

\title{Solving the 2D Burgers Equation: Physical Simulation vs NFTM}
\author{Samuel, Lucía and Alexandra}
\date{17th October 2025}

% The next block of commands puts the table of contents at the beginning of each section and highlights the current section
% \AtBeginSection[]
% {
%   \begin{frame}
%     \frametitle{Table of Contents}
%     \tableofcontents[currentsection]
%   \end{frame}
% }



\begin{document}

% Make the title page
% \frame{\titlepage}

\begin{frame}
\frametitle{Project Overview}
\begin{itemize}
  \item Goal: Compare a physical simulator and a Neural Field Turing Machine (NFTM) for solving the 2D Burgers equation.
  \item Approach: Build a numerical solver, then train and evaluate NFTM on the same problem.
\end{itemize}
\end{frame}


\begin{frame}
\frametitle{2D Burgers Equation}
\begin{align*}
  u_t + u u_x + v u_y &= \nu (u_{xx} + u_{yy}) \\
  v_t + u v_x + v v_y &= \nu (v_{xx} + v_{yy})
\end{align*}
\begin{itemize}
  \item $u(x, y, t)$, $v(x, y, t)$: velocity components
  \item $\nu$: viscosity
\end{itemize}
\end{frame}


\begin{frame}
\frametitle{Physical System: Numerical Simulation}
\begin{itemize}
  \item Implements the PDE directly using finite difference methods.
  \item Discretizes space and time; updates velocity fields step by step.
  \item Parameters: grid size, time step, viscosity, initial/boundary conditions.
  \item Produces ground-truth data for model comparison.
\end{itemize}
\end{frame}


\begin{frame}
\frametitle{NFTM: Neural Field Turing Machine}
\begin{itemize}
  \item Learns to update field values using neural controllers and local memory patches.
  \item Trained on data from the physical simulator or existing datasets.
  \item Can generalize to new initial conditions or parameters.
\end{itemize}
\end{frame}



\begin{frame}
\frametitle{Datasets}
\begin{itemize}
  \item Use existing benchmark datasets for PDEs (if available).
  \item Generate custom datasets using our physical simulator (varied initial/boundary conditions, parameters).
  \item Enables robust training and fair comparison of NFTM and physical solver.
\end{itemize}
\end{frame}


\begin{frame}
\frametitle{Summary \& Next Steps}
\begin{itemize}
  \item Build and validate the physical simulator.
  \item Train NFTM on generated and existing datasets.
  \item Compare performance, accuracy, and generalization.
\end{itemize}
\end{frame}

\end{document}
